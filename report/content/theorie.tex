\section{Theoretische Perspektiven}

Das Player Piano Projekt an der Stanford University (\textit{SUPRA}) wurde 2014 ins Leben gerufen um die umfangreiche Sammlung von Notenrollen in der Bibliothek von Stanford Forscher:innen und anderen interessierten zugänglich zu machen \parencite*[]{shi_2019}.
Aktuell befinden sich circa 20.000 Notenrollen in der Sammlung in Stanford von denen gut 1000 digitalisiert wurden \parencite*[]{broadwell_2022}.
Das Projekt fokussierte sich dabei zunächst auf die digitalisierung von Notenrollen des Typs Welte T-100\footnote{Auch rote Welte Rollen genannt, da sie meist auf roten Papiert produziert wurden.}, die erste Form der reproduzierenden Notenrolle.

Für die digitalisierung wurde zunächst ein Gerät konstruiert, dass es erlaubt hochauflösende\footnote{TIFF mit 300dpi Auflösung.} Bilder der Notenrollen zu erstellen.
Um eine hochwertige Emulation der Originalrollen zu ermöglichen wurden zunächst einige Fehlerkorrekturen vorgenommen.
Es wurde etwa eine Korrektur zur begradigung der Notenrollenscans eingesetzt um die, grade in den späteteren Teilen der Rollen, auftretenden Verzerrungen zu korrigieren.
Weiterhin wurden Löcher die uncharacteristisch\footnote{Beispielsweise von Form oder Orientierung} und vermutlich auf Beschädigungen zurückzuführen sind herausgefiltert und schnell aufeinanderfolgende Löcher die als einzelne Note gespielt werden\footnote{Sogenannte \textit{Bridges}.} zusammengefasst \parencite*[519f]{shi_2019}.

Um nun eine möglichst originalgetreue emulation eines Originalabspielgerätes zu erreichen untersuchten \textcite[521f]{shi_2019} die Funktion der vorhandenen Steuerungsfuntktionen auf den Welte Rollen mithilfe von Originaldokumenten, Testrollen und weiteren Quellen.
Die so decodierten Informationen wendeten sie dann bei der Erstellung der Midi Dateien und in folge für die daraus generierten Audio Repräsentationen an.
Es wurden schlussendlich noch einige kleinere Korrekturen angewendet wie eine korrektur der bauartbedingte Tempsteigerung die beim abspielen der Originalrollen typisch war bevor die erzeugten Midis durch ein generisches Software-Piano gerendert wurden.

