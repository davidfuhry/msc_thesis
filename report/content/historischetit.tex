\FloatBarrier
\section{Historische Toninformationsträger}

\begin{figure*}[t]
    \centering
    \includegraphics[width=\textwidth]{graphics/placeholder.png}
    \caption{Lochplatten zweier verschiedener Typen aus dem Bestand des Musikinstrumentenmuseums der Universität Leipzig}
    \label{platten}
\end{figure*}

Als Toninformationsträger\footnote{Auch Historical Music Storage Media.} werden die Medien bezeichnet, die zum Betrieb von mechanischen Musikinstrumenten benötigt werden.
Eine Definition für meschanische Musikinstrumente ist etwa in \textcite[]{mgg_mechanische} gegeben:

\begin{quotation}
    Mechanische Musikinstrumente (selbstspielende Musikinstrumente, Musikautomaten, Automatophone) sind Musikinstrumente, bei denen die Tonfolge selbsttätig durch einen Toninformationsträger gesteuert wird. Die Mehrzahl der mechanischen Musikinstrumente reproduziert Musik ohne Einwirkungsmöglichkeit des Menschen. \parencite[]{mgg_mechanische}
\end{quotation}

Die Geschichte solcher meschnischen Musikinstrumente reicht zurück bis in die Antike, in der bereits einfache Orgeln existierten die über das setzen von Stiften auf einer sich drehenden Walze gesteuert werden konnten.
Ihren Höhepunkt erreichten diese Instrumente während des späten 19. und frühen 20. Jahrhunderts als eine Vielzahl von verschiedenen, immer komplexeren mechanischen Instrumenten konstruiert wurden \parencite[10ff]{bowers_1972}.
Mit der Entwicklung von besseren und günstigeren Methoden zur Audioaufnahme und -reproduktion in den 1930ern wurden die mechanischen Musikinstrumente praktisch komplett vom Markt verdrängt \parencite[2]{zoltan_1994}.
Für eine ausführliche Übersicht über die Geschichte der mechanischen Musikinstrumente sei auf die entsprechende Literatur verwiesen, etwa \textcite[]{bowers_1972,bowers_1975} sowie \textcite[]{mgg_mechanische}.

Im weiteren sollen vorrangig die für diese Arbeit relevanten Arten und Formate von Toninformationsträgern weiter betrachtet werden.
% Platten: 1882 bis 1910, Rollen 1900 bis 1930
Diese wurden im Zeitraum zwischen 1882 und 1930 produziert und lassen sich in zwei Kategorien unterteilen: Plattenförmige sowie Rollenförmige Medien, wobei die Plattenförmigen Medien sowohl zeitlich als auch technologisch als Vorgänger der Rollen betrachtet werden können.

\subsection{Plattenförmige Toninformationsträger}

Solche Plattenförmigen Medien wurden ab 1875 produziert und stellten einen paradigmenwechsel in der Konstruktion mechanischer Musikinstrumente dar.
Waren bei früheren Modellen die Toninformationen direkt im Instrument Kodiert (beispielsweise über Walzen), wurden sie nun seperat auf Platten kodiert und waren somit unabhängig vom Instrument.
Eine solche Platte hat zumeist eine Laufzeit von einer kleinen einstelligen Zahl von Minuten.
Damit wurde es erstmals Möglich eine größere Sammlung von Musikstücken zu besitzen \parencite[]{mgg_mechanische}.

Die frühen Modelle solcher Plattenbasierten mechanischen Instrumente nutzten Platten aus Pappe auf denen die Toninformationen kodiert waren.
Einige solcher Platten sind in Abbildung \ref{platten} zu sehen.
Diese Platten haben mehrere Spuren, jede Spur ist einem bestimmten Ton zugeordnet, meist vom innenliegenden Bass aufsteigend sortiert.
Die Töne selbst wurden mittels Löchern binär in die Platten kodiert, dort wo ein Loch vorhanden ist klingt der Ton der entsprechenden Spur.
Wichtig anzumerken ist, dass die meisten Formate nicht chromatisch angelegt waren, die sehr populären Platten der Marke Ariston beispielsweise haben 24 Spuren decken aber über 3 Oktaven ab.\footnote{Siehe \href{https://musixplora.de/mxp/2003520}{musixplora.de/mxp/2003520}.}
Durch das auslassen bestimmter Töne konnten zwar nicht alle Stücke in ihren Origantonarten kodiert werden, es konnte aber ein größeres tonales Spektrum abgebildet werden ohne die Medien und Abspielgeräte in gleichem Maße zu vergrößern.

\begin{figure*}[t]
    \centering
    \includegraphics[width=\textwidth]{graphics/placeholder.png}
    \caption{Abspielgerät für Lochplatten der Marke Ariston aus dem Bestand des Musikinstrumentenmuseums der Universität Leipzig}
    \label{aristonplayer}
\end{figure*}

Die Abspielgeräte für solche Platten nutzen ein pneumatisches Verfahren.
Die Platte wurde dabei in ein Abspielgerät eingelegt das über eine Leiste mit Zungen verfügte die an den Tonspuren der Platte ausgerichtet waren.
Zum Abspielen wurde die Platte, je nach Gerät per Handkurbel oder automatisiert, Gedreht und über Luft durch die Zungen bewegt.
An den Stellen an denen Löcher vorhanden waren konnte die Luft durch die Zungen passieren, so dass ein Ton erzeugt wurde, ähnlich einer Mundharmonika.
Der so entstehende Klang dieser sogenannten Organetten wurde jedoch häufig als nicht zufriedenstellend empfunden \parencite[]{mgg_mechanische}.

Eine daraus resultierende Entwicklung war der mechanische Klaviervorsetzer.
Dieser nutzte Platten die denen für die Organetten sehr ähnlich waren und Informationen nach dem selben Prinzip kodierten.
Es wurden damit jedoch keine Tonerzeuger direkt angesteuert, stattdessen besaß das Gerät eine Reihe mechanischer Finger.
Vor einem Klavier platziert spielte das Gerät so die auf der Platte kodierten Informationen auf diesem.
Bilder einer Ariston Orgranette sowie eines solchen mechanischen Klaviervorsetzers befinden sich in Abbildung \ref{aristonplayer}.

Eine weitere Weiterentwicklung der frühen Organetten waren Platten aus Metall.
Bei diesen wurden kleine Teile des Blechs umgebogen, so dass Noppen entstanden die in der Lage waren Tonkämme einzureißen.
Die Abspielgeräte für solche Platten verfügten häufig über Zusatzinstrumente wie Triangeln oder Trommeln und waren unter anderem in Wirtshäusern sehr populär \parencite[]{mgg_mechanische}.

Während die Digitalisierung solcher Metallplatten teil des DISKOS Projektes ist, ist sie aufgrund der fundamental unterschiedlichen Art der Informationskodierung solcher Platten kein Teil dieser Arbeit.
Insgesamt wurden im Rahmen des DISKOS Projektes Bilder von 100 Pappplatten aus dem Bestand des Musikinstrumentenmuseums angefertigt, davon 60 Platten des Typs Ariston und 40 Platten für einen mechanischen Klaviervorsetzer, welche mit der in dieser Arbeit vorgestellten Softwarelösung digitalisiert werden.

\subsection{Notenrollen}

\begin{figure*}[t]
    \centering
    \includegraphics[width=\textwidth]{graphics/placeholder.png}
    \caption{Abspielgerät für Lochplatten der Marke Ariston aus dem Bestand des Musikinstrumentenmuseums der Universität Leipzig}
    \label{pianoroll}
\end{figure*}

Während Plattenförmige Toninformationsträger eine signifikante Weiterentwicklung gegenüber den früheren, walzenbasierten mechanischen Musikinstrumenten darstellten, wurde mit der Klavierrolle ein weiterer Entwicklungsschritt gemacht.
Dieses um die Jahrhunderwende eingeführte Format unterscheidet sich zum einen in der Form der Toninformationsträger, die nun als mehrere Meter lange Rollen ausgeführt wurden, aber vor allem durch die eingesetzten Techniken in den Abspielgeräten.
Ein Abbild einer solchen Rolle ist Abbilgund \ref{pianoroll} zu entnehmen.

Die Veränderung der Toninformationsträger selbst ist, abgesehen von der neuen Form die eine Spielzeit von bis zu 15 Minuten ermöglichte, nur relativ gering.
Eine Rolle ist analog zu den früheren Pappplatten in mehrere Spuren unterteilt, die nun allerdings nicht mehr Kreisförmig sondern parallel, meist über die Rollenbreite von Bass (Links) nach Sopran (Rechts) sortiert, angelegt sind.
Die Spuren enthalten weiterhin Löcher die die Toninformationen auf der Platte kodieren, ein wesentlicher Unterschied zu Pappplatten besteht in der Kodierung länger zu spielender Töne: Diese werden nun nicht mehr durch eine durchgehende Lochung sondern durch in sehr kurzem Abstand platzierte, z.T. auch überlappende, Löcher kodiert.

Frühe Abspielgeräte funktionierten analog zu mechanischen Klaviervorsetzern für Pappplatten als Vorsetzer und unterschieden sich im wesentlichen durch die längere Laufzeit der Rollen gegenüber Platten.
Der entscheidende Vorteil dieses Systems lag in der pneumatischen Steuerung des Abspielgeräts.
Zum Abspielen wurde die Rolle in das Abspielgerät eingelegt und dieses erzeugte, bei den ersten Geräten durch menschliche Kraft, ein Vakuum über die gesamte Rollenbreite.
Erreichte die Rolle nun eine Stelle an der ein Loch vorhanden war, konnte die Luft durch das Loch strömen und so eine pneumatische Steuerung betreiben.
Diese erlaubte es durch die Stärke des angelegten Vakuums auch die Stärke des Anschlags auf dem bespielten Klavier zu variieren \parencite[]{mgg_mechanische}.

Mit dieser Technik wurden kurz vor der Jahrhundertwende die ersten selbstspielenden Klaviere\footnote{Auch Player Pianos.} konstruiert.
Die meisten dieser Klaviere verfügten über 2 seperate Windladen für die Erzeugung des Vakuums und konnten so die Lautstärke von Bass und Diskant getrennt regulieren.
Während die Variation der Vakuumstärke bei den früheren Modellen, genau wie die Wahl der Geschwindigkeit, noch durch die benutzende Person erfolgte wurden schnell zunehmend komplexere Modelle entwickelt die diese Aufgaben übernahmen \parencite[]{mgg_mechanische}.

Im Jahr 1904 patentierte die Firma Welte mit dem Welte-Mignon-System das erste Reproduktionsklavier\footnote{Auch Reproducing Player Piano.} ein System, das komplett ohne menschliche Eingaben Musikstücke wiedergeben konnte \parencite[]{mgg_mechanische}.
Um dies zu ermöglichen wurden Informationen beispielsweise zu Dynamik und Pedalbenutzung auf dafür vorgesehenen Spuren der Notenrolle kodiert.
Die Rollen des Welte-Mignon-Systems etwa besaßen 100 Spuren von denen nur 80 Tönen zugeordnet waren, die übrigen 20 dienten zur Steuerung des Abspielgerätes.
Solche Informationen konnte im Herstellungsprozess entweder manuell kodiert werden oder aber durch die Aufnahme von Stücken über speziell dafür konstruierte Aufnahmeklaviere.
An diesen konnte ein:e Künstler:in ein Stück einspielen welches danach in einem mehrstufigen Prozess durch den Rollenhersteller auf eine Notenrolle übertragen wurde.
Das genaue Verfahren mit dem dies Geschah wurde durch die Hersteller mit größter Geheimhaltung geschützt und da keines der Aufnahmegeräte bis heute überdauert hat kann darüber nur spekuliert werden \parencite[]{zoltan_1994}.

Da Audioaufnahmen zur damaligen Zeit noch nicht in vergleichbarer Zeit möglich waren, bescherten grade diese eigens eingespielten Künstlerrollen den Reproduktionsklavieren eine enorme Popularität.
Schnell entwickelte sich eine Vielzahl von verschiedenen Systemen von Reproduktionsklavieren und dazugehörigen Toninformationsträgern und in den 1920er Jahren hatte die Mehrzahl der bekannten Pianisten für einen oder mehrere der Hersteller Notenrollen eingespielt \parencite[]{bowers_1972}.

Im Rahmen des TASTEN Projekts wurden zwischen 2018 und 2020 am Musikinstrumentenmuseum der Universität Leipzig gut 5000 Notenrollen als Bilder digitalisiert.
Von diesen liegen ca. 3000 als vollständige Digitalisate vor, von ca. 2000 der Rollen sind nur die Anfänge der Rolle, auf denen Metainformationen wie Titel, Format, etc. vermerckt sind, digitalisiert.
Digitalisiert wurden über 30 verschiedene Formate von Notenrollen, darunter auch 358 Rollen des Types Welte Mignon.
Während Ziel des DISKOS Projektes ist Notenrollen aller vorliegenden Formate verarbeiten zu können, beschränkt sich diese Arbeit auf Rollen des Typs Hupfeld Phonola Solodant, einem verhältnismäßig einfach Format mit 77 Spuren.