\FloatBarrier
\section{Historische Toninformationsträger}

\begin{figure*}[t]
    \centering
    \includegraphics[width=\textwidth]{graphics/placeholder.png}
    \caption{Lochplatten zweier verschiedener Typen aus dem Bestand des Musikinstrumentenmuseums der Universität Leipzig}
    \label{platten}
\end{figure*}

Als Toninformationsträger\footnote{Auch Historical Music Storage Media.} im Sinne dieser Arbeit werden die Medien bezeichnet, die zum Betrieb von mechanischen Musikinstrumenten benötigt werden.
Eine Definition für mechanische Musikinstrumente ist etwa in \textcite[]{mgg_mechanische} gegeben:

\begin{quotation}
    Mechanische Musikinstrumente (selbstspielende Musikinstrumente, Musikautomaten, Automatophone) sind Musikinstrumente, bei denen die Tonfolge selbsttätig durch einen Toninformationsträger gesteuert wird. Die Mehrzahl der mechanischen Musikinstrumente reproduziert Musik ohne Einwirkungsmöglichkeit des Menschen. \parencite[I. Definition]{mgg_mechanische}
\end{quotation}

Die Geschichte solcher mechanischen Musikinstrumente reicht zurück bis in die Antike, in der bereits einfache Orgeln existierten die über das setzen von Stiften auf einer sich drehenden Walze gesteuert werden konnten.
Den Höhepunkt ihrer Popularität erreichten sie während des späten 19. und frühen 20. Jahrhunderts, als eine Vielzahl von verschiedenen, immer komplexeren mechanischen Instrumenten konstruiert wurde \parencite[10-12]{bowers_1972}.
Mit der Entwicklung von besseren und günstigeren Methoden zur Audioaufnahme und -reproduktion in den 1930ern wurden die mechanischen Musikinstrumente praktisch komplett vom Markt verdrängt \parencite[2]{zoltan_1994}.
Für eine ausführliche Übersicht über die Geschichte der mechanischen Musikinstrumente sei auf die entsprechende Literatur verwiesen, etwa \textcite[]{bowers_1972,bowers_1975} sowie \textcite[]{mgg_mechanische}.

Die Medien für solche mechanischen Instrumente kodieren Informationen aus denen das mechanische Instrument Musik erzeugen kann.
Üblicherweise werden dafür zumindest grundlegende Informationen wie der Beginn, die Länge sowie die zugehörige Note für jeden Ton auf dem Medium kodiert.
Es existieren aber auch Instrumente und mit ihnen Formate von Toninformationsträgern die zusätzlich weitere Informationen zu Dynamik und Spielweise kodieren können.
Im weiteren sollen vorrangig die für diese Arbeit relevanten Typen und Formate von Toninformationsträgern weiter betrachtet werden.
% Platten: 1882 bis 1910, Rollen 1900 bis 1930
Diese wurden im Zeitraum zwischen 1882 und 1930 produziert und lassen sich grob in zwei Kategorien unterteilen: Plattenförmige sowie rollenförmige Medien, wobei die plattenförmigen Medien sowohl zeitlich als auch technologisch als Vorgänger der Rollen betrachtet werden können.

\subsection{Plattenförmige Toninformationsträger}

Plattenförmigen Medien wurden ab ca. 1875 produziert und stellten einen paradigmenwechsel in der Konstruktion mechanischer Musikinstrumente dar.
Waren bei früheren Modellen die Toninformationen direkt im Instrument kodiert (beispielsweise über Walzen), wurden sie nun separat auf Platten kodiert und waren somit erstmals unabhängig vom Instrument.
Eine solche Platte hat zumeist eine kurze Laufzeit von einer kleinen einstelligen Zahl von Minuten und konnte damit zumindest kürzere Stücke oder Motive aus größeren Kompositionen enthalten.
Durch diese Entwicklung wurde es erstmals Möglich eine größere Sammlung von Musikstücken zu besitzen \parencite[III.5.c. Plattenspieldosen und Drehinstrumente]{mgg_mechanische}.

Frühe Formate dieser Form von mechanischen Instrumente nutzten Platten aus Pappe auf denen die Toninformationen in Form von Löchern kodiert sind (im Folgenden Lochplatten genannt).
Lochplatten dieser Formate sind sich grundsätzlich recht ähnlich und unterscheiden sich vor allem in den verfügbaren Tönen sowie den physischen Abmessungen der Medien.
Beispiele für zwei Lochplatten unterschiedlicher Formate dieses Typs sind in Abbildung \ref{platten} zu sehen.

Die Lochplatten haben mehrere Spuren, jede Spur ist einem bestimmten Ton zugeordnet, meist sind die Spuren aufsteigend nach Tonhöhe sortiert, mit der tiefsten Note auf der innersten Spur.
Die Töne selbst wurden mittels Löchern binär in die Platten kodiert, dort wo ein Loch vorhanden ist klingt der Ton der entsprechenden Spur.
Wichtig anzumerken ist, dass die meisten Formate nicht chromatisch angelegt waren, die populären Platten der Marke Ariston beispielsweise haben 24 Spuren decken aber über 3 Oktaven ab.\footnote{Siehe \textcite[]{mxp_2003520} für eine genaue Aufschlüsselung des Tonvorrats dieses Formates.}
Durch das auslassen bestimmter Töne ergaben sich zwar Einschränkungen bei der Kodierung von musikalischen Stücken auf die Medien, es konnte aber ein größerer Tonumfang abgebildet werden ohne die Medien und Abspielgeräte in unpraktikable Größen zu skalieren.

\begin{figure*}[t]
    \centering
    \includegraphics[width=\textwidth]{graphics/placeholder.png}
    \caption{Abspielgerät für Lochplatten der Marke Ariston aus dem Bestand des Musikinstrumentenmuseums der Universität Leipzig}
    \label{aristonplayer}
\end{figure*}

Die Abspielgeräte für solche Platten (siehe Abbildung \ref{aristonplayer}) nutzen Verfahren zur Tonerzeugung, dass dem einer Mundharmonika nicht unähnlich ist.
Eine Lochplatte wurde in ein Abspielgerät eingelegt das über eine Leiste mit Zungen verfügt die den gesamten Radius des informationstragenden Teils der Lochplatte bedeckt.
Zum Abspielen wurde die Lochplatte, je nach Gerät per Handkurbel oder automatisiert, Gedreht und Luft wurde durch die Zungen bewegt.
An den Stellen an denen Löcher vorhanden waren konnte die Luft durch die Zungen passieren, so dass ein Ton erzeugt wurde.
Der so entstehende Klang dieser sogenannten Organetten wurde jedoch häufig als nicht zufriedenstellend empfunden \parencite[III.5.c. Plattenspieldosen und Drehinstrumente]{mgg_mechanische}.

Eine der Entwicklungen die gemacht wurden um dieses Defizit im Klang auszugleichen war der mechanische Klaviervorsetzer.
Dieser nutzt Platten die denen für die Organetten sehr ähnlich sind und Informationen nach dem selben Prinzip kodierten.
Im Gegensatz zu diesen war im Abspielgerät selbst (siehe Abbildung \ref{aristonplayer}) allerdings keine Tonerzeugung integriert.
Stattdessen besaß das Gerät eine Reihe mechanischer Finger, die, vor einem Klavier platziert, das auf der Lochplatte kodierte Stück auf diesem "spielten".

Eine andere Weiterentwicklung der frühen Organetten stellen plattenförmige Medien aus Metall dar.
Bei der Produktion solcher wurden kleine Teile des Blechs umgebogen, so dass Noppen entstanden die in der Lage waren Tonkämme anzureißen.
Die Abspielgeräte für solche Platten verfügten häufig über Zusatzinstrumente wie Triangeln oder Trommeln und waren unter anderem in Wirtshäusern sehr populär \parencite[III.5.c. Plattenspieldosen und Drehinstrumente]{mgg_mechanische}.
Während die Digitalisierung solcher Metallplatten Teil des DISKOS Projektes ist, ist sie aufgrund der fundamental unterschiedlichen Art der Informationskodierung auf diesen Medien kein Teil dieser Arbeit.

Im Rahmen des DISKOS-Projektes wurden Bilder eines vollständiges Konvolut aus 60 Lochplatten der Marke Ariston angefertigt die mit der in dieser Arbeit vorgestellten Software digitalisiert werden.
Weitere 40 Lochplatten für einen Klaviervorsetzer befinden sich noch im Fotografieprozess, konnten aber bereits für Testdigitalisierungen genutzt werden.

\subsection{Notenrollen}

\begin{figure*}[t]
    \centering
    \includegraphics[width=\textwidth]{graphics/placeholder.png}
    \caption{Abspielgerät für Lochplatten der Marke Ariston aus dem Bestand des Musikinstrumentenmuseums der Universität Leipzig}
    \label{pianoroll}
\end{figure*}

Während die mit der Lochplatte eingeführte Trennung von Abspielgerät und Toninformationen eine signifikante Weiterentwicklung gegenüber den früheren, walzenbasierten mechanischen Musikinstrumenten darstellte, wurde mit der Notenrolle\footnote{Auch Klavierrolle genannt.} ein weiterer Entwicklungsschritt gemacht.
Dieses um die Jahrhundertwende eingeführte Format unterscheidet sich zum einen in der Form der Toninformationsträger, die nun als mehrere Meter lange Rollen ausgeführt wurden, aber vor allem durch die eingesetzten Techniken in den Abspielgeräten.
Ein Abbild einer solchen Rolle ist Abbildung \ref{pianoroll} zu entnehmen.

Die Veränderungen an der Art der Informationskodierung auf den Notenrollen im Vergleich zu Lochplatten ist relativ klein.
Zunächst unterscheiden sich die Notenrollen vor allem durch die neue physische Form die eine Spielzeit von bis zu 15 Minuten ermöglichte.
Eine Notenrolle ist analog zu den früheren Lochplatten in mehrere Spuren unterteilt, die nun allerdings nicht mehr kreisförmig, sondern parallel, meist über die Rollenbreite von Bass (Links) nach Diskant (Rechts) sortiert, angelegt sind.
Die Spuren enthalten weiterhin Löcher die über ihre Position die Toninformationen auf der Notenrolle kodieren.
Ein wesentlicher Unterschied zu Pappplatten besteht in der Kodierung länger zu spielender Töne: Diese werden nun nicht mehr durch eine durchgehende Lochung sondern durch in sehr kurzem Abstand platzierte, z.T. auch überlappende, Löcher kodiert.

Frühe Abspielgeräte funktionierten, analog zu mechanischen Klaviervorsetzern für Pappplatten, als Vorsetzer und unterschieden sich im wesentlichen durch die längere Laufzeit der Notenrollen.
Der entscheidende Vorteil dieses Systems lag in der pneumatischen Steuerung des Abspielgeräts.
Zum Abspielen wird die Rolle in das Abspielgerät eingelegt und dieses erzeugte, bei den ersten Geräten durch menschliche Kraft, später durch Elektromotoren, ein Vakuum über die gesamte Rollenbreite.
Erreichte die Rolle nun eine Stelle an der ein Loch vorhanden war, kann die Luft durch das Loch strömen und so eine pneumatische Steuerung betreiben.
Diese erlaubte es über die Stärke des angelegten Vakuums auch die Stärke des Anschlags auf dem bespielten Klavier zu variieren \parencite[III.5.d. Selbstspielende Klaviere]{mgg_mechanische}.

Mit dieser Technik wurden kurz vor der Jahrhundertwende die ersten selbstspielenden Klaviere\footnote{Auch Player Pianos genannt.} konstruiert.
Die meisten dieser Klaviere verfügten über 2 separate Windladen für die Erzeugung des Vakuums und konnten so die Lautstärke von Bass und Diskant getrennt regulieren.
Während die Variation der Vakuumstärke bei den früheren Modellen, genau wie die Wahl der Geschwindigkeit, noch durch die benutzende Person erfolgte, wurden schnell zunehmend komplexere Abspielgeräte und Rollenformate entwickelt bei denen diese Aufgabe vom Abspielgerät selbst übernommen wurde \parencite[III.5.d. Selbstspielende Klaviere]{mgg_mechanische}.

Im Jahr 1904 patentierte die Firma Welte mit dem Welte-Mignon-System das erste Reproduktionsklavier\footnote{Auch Reproducing Player Piano genannt.} ein System, das komplett ohne menschliche Eingaben Musikstücke wiedergeben konnte \parencite[]{mgg_mechanische}.
Um dies zu ermöglichen wurden Informationen beispielsweise zu Dynamik und Pedalbenutzung auf dafür vorgesehenen Spuren der Notenrolle kodiert.
Die Rollen des Typs Welte-Mignon Welte T100 etwa besaßen 100 Spuren von denen nur 80 zu Tönen zugeordnet waren, die übrigen 20 dienten zur Steuerung des Abspielgerätes \parencite[]{mxp_2002522}.

Solche Informationen konnte im Herstellungsprozess entweder manuell kodiert werden oder aber durch die direkte Aufnahme von Stücken über speziell dafür konstruierte Aufnahmeklaviere erzeugt werden.
An diesen konnten Künstler:innen ein Stück einspielen welches danach in einem mehrstufigen Prozess durch den Rollenhersteller auf eine Notenrolle übertragen wurde.
Das genaue Verfahren mit dem dies Geschah wurde durch die Hersteller mit größter Geheimhaltung geschützt und da keines der Aufnahmegeräte bis heute überdauert hat kann darüber nur spekuliert werden \parencite[]{zoltan_1994}.

Da Audioaufnahmen zur damaligen Zeit noch nicht in vergleichbarer Zeit möglich waren, bescherten grade diese eigens eingespielten Künstlerrollen den Reproduktionsklavieren eine enorme Popularität.
Schnell entwickelte sich eine Vielzahl von verschiedenen Systemen von Reproduktionsklavieren und dazugehörigen Toninformationsträgern und in den 1920er Jahren hatte die Mehrzahl der bekannten Pianist:innen für einen oder mehrere der Hersteller Notenrollen eingespielt \parencite[]{bowers_1972}.

Im Rahmen des vorangegangenen TASTEN-Projekts wurden zwischen 2018 und 2020 am Musikinstrumentenmuseum der Universität Leipzig ca. 5000 Notenrollen in Form von Bildern digitalisiert.
Von diesen liegen ca. 3000 als vollständige Scans vor, von dem restlichen ca. 2000 der Rollen sind nur die Anfänge der Rolle, auf denen Metainformationen wie Titel, Format, etc. vermerkt sind, gescannt worden.
Das gesamte Konvolut umfasst über 30 verschiedene Formate von Notenrollen.
Während im Rahmen des DISKOS-Projektes aus den Scans von Notenrollen aller vorliegenden Formate Midi-Dateien generiert werden sollen, beschränkt sich diese Arbeit auf Rollen des Typs Hupfeld Phonola Solodant, einem verhältnismäßig einfach Format mit 77 Spuren.