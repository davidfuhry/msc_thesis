\section{Einleitung}

Historische Medien wie Lochplatten und Klavierrollen sind die ältesten, kommerziell in größerer Stückzahl produzierten und auch für den Privatgebrauch gedachten Toninformationsträger.
Sie wurden vom Ende des 19. Jahrhunderts bis zur frühen Mitte des 20. Jahrhunderts von einer Vielzahl von Musikstücken, durch verschiedene Hersteller und in einer hohen Zahl unterschiedlicher Formate produziert.
Die Medien sind zeithistorische Zeugnisse aus einer Zeit, in der privater Musikkonsum erstmals einer breiteren Gruppe von Personen möglich wurde.

Neben durch die Hersteller auf der Basis von Kompositionen erstellten Medien, existieren mit den sogenannte Player Rolls auch Notenrollen, die an einem speziell modifizierten Klavier von kontemporären Pianist:innen eingespielt wurden.
Diese Notenrollen bilden die einzigen Aufnahmen vieler der bekanntesten Pianist:innen aus einer Zeit, in der Audioaufnahmen technisch noch nicht ausgereift genug für die Aufnahme musikalischer Darbietungen waren.

Diese Eigenschaften machen solche historischen Toninformationsträger zu einer wichtigen Ressource für die moderne musikwissenschaftliche Forschung.
Während eine signifikante Anzahl von Medien, insbesondere Notenrollen, bis heute überlebt hat, wird sich ihr Zustand auf absehbare Zeit weiter verschlechtern und so zu irreversiblen Verlusten von Medien und den darauf enthaltenen Informationen führen.

Um diese zeithistorischen Dokumente auch in Zukunft für die musikwissenschaftliche Forschung zugänglich zu machen und gleichzeitig weitere Schäden an den Medien, etwa durch Abnutzung, zu minimieren, bietet sich die Digitalisierung ebendieser an.
Dazu zählt zum einen die Erstellung von hochauflösenden Bildern der Medien, zum anderen aber auch die Digitalisierung der auf den Medien kodierten Toninformationen.
Verfahren zu diesem Zweck existieren, sind aber überwiegend schlecht dokumentiert und nicht problemlos zugänglich (siehe Abschnitt \ref{Forschungsstand}).

Das Musikinstrumentenmuseum der Universität Leipzig ist im Besitz einer umfangreichen Sammlung von solchen historischen Toninformationsträgern.
Im Rahmen des vergangenen Forschungsprojektes TASTEN wurden von ca. 3000 der Notenrollen aus dieser Sammlung vollständige Scans erstellt.

Ziel des aktuellen Forschungsvorhabens DISKOS an der Forschungsstelle des Musikinstrumentenmuseums ist die Digitalisierung dieser Notenrollen und weiterer, plattenförmiger Toninformationsträger aus derselben Sammlung, sowie die inhaltliche Analyse der digitalisierten Medien.
Eine kurze Übersicht dazu findet sich bei \textcite[]{khulusi_2022}.
Zur Umsetzung der Ziele des Forschungsprojekts soll unter anderem eine Anwendung entwickelt werden, die Medien aller Formate in der Sammlung verarbeiten kann.
Die Anwendung soll auch für die inhaltlichen Analysen nutzbar sein.

Zu diesem Zweck wurde die in dieser Arbeit beschriebene Software implementiert.
Sie ist in der Lage, aus den Bilddaten von historischen Toninformationsträgern, die darauf kodierten musikalischen Informationen zu extrahieren.

Es wird zunächst ein voll funktionsfähiges Verfahren zur Digitalisierung von Lochplatten vorgestellt.
Ebenso wird eine erste Implementierung für ein Verfahren zur Digitalisierung von Notenrollen eingeführt.
Die Anwendung gibt nach der Verarbeitung der Bilddaten eine standardisierte Midi-Datei aus.

Im weiteren werden die in die Anwendung integrierten Analyse- und Visualisierungstools, die in Zusammenarbeit mit den am Forschungsprojekt beteiligten Musikwissenschaftler:innen zur Beantwortung musikwissenschaftlicher Fragestellungen entwickelt wurden, demonstriert.

