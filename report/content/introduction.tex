\section{Einleitung}

Historische Medien wie Lochplatten und Klavierrollen sind die ältesten kommerziell in größerer Stückzahl produzierten und auch für den Privatgebrauch gedachten Toninformationsträger.
Sie wurden vom Ende des 19. Jahrhunderts bis zur frühen Mitte des 20. Jahrhunderts von einer Vielzahl von Musikstücken von verschiedenen Herstellern in einer hohen Zahl verschiedener Formate produziert.
Die Medien sind Zeugnisse aus einer Zeit in der privater Musikkonsum erstmals einer breiteren Gruppe von Personen möglich wurde.
Neben durch die Hersteller aus Kompositionen erstellten Medien existieren auch von sogenannten Player Rolls, Notenrollen die an einem speziell modifizierten Klavier von Pianisten der Zeit eingespielt wurden.
Solcher Notenrollen bilden die einzigen Aufnahmen vieler der bekanntesten Pianisten aus einer Zeit in der Audioaufnahmen technisch noch nicht ausgereift genug für die Aufnahme musikalischer Darbietungen waren.

Diese Eigenschaften machen solche historischen Toninformationsträger zu einer unschätzbaren Resource für die moderne musikwissenschaftliche Forschung.
Während eine signifikante Menge der Medien, insbesondere von Notenrollen, bis heute überlebt hat, wird sich ihr Zustand auf absehbare Zeit weiter verschlechtern und so zu irreversiblen Verlusten von Medien und den darauf Enthaltenen Informationen führen.
Um diese zeithistorischen Dokumente auch in Zukunft für die musikwissenschaftliche Forschung zugänglich zu machen und gleichzeitig weitere Schäden an den Medien durch Abnutzung zu minimieren bietet sich die Digitalisierung der selbigen an.
Dazu zählt zum einen die Erstellung von hochauflösenden Bildern der Medien, um diese möglichst originalgetreu abzubilden, zum anderen aber auch die Digitalisierung der in den Medien kodierten Toninformationen.
Verfahren zu diesem Zweck existieren, sind aber zu überwiegend schlecht dokumentiert und der Öffentlichkeit nicht problemlos zugänglich (siehe Abschnitt \ref{Forschungsstand}).

Das Musikinstrumentenmuseum der Universität Leipzig ist im Besitz einer umfangreichen Sammlung von solchen historischen Toninformationsträgern.
Im Rahmen des vergangenen Forschungsprojektes TASTEN wurden von ca. 3000 der Notenrollen aus dieser Sammlung Scans erstellt.
Ziel des aktuellen Forschungsvorhabens DISKOS an der Forschungsstelle des Musikinstrumentenmuseums ist die Digitalisierung dieser Notenrollen und weiterer, plattenförmiger Toninformationsträger aus derselben Sammlung sowie die inhaltliche Analyse der digitalisierten Medien.
Eine kurze Übersicht dazu findet sich bei \textcite[]{khulusi_2022}.
Zur Umsetzung der Ziele des Forschungsprojektes soll unter anderem eine Anwendung entwickelt werden die Medien aller sich im Bestand des Musikinstrumentenmuseums befindlichen Formate verarbeiten kann.
Die Anwendung soll auch für die späteren inhaltlichen Analysen nutzbar sein.

Zu diesem Zweck wurde die in dieser Arbeit beschriebene Software implementiert.
Sie ist in der Lage aus den Bilddaten von historischen Toninformationsträgern die darauf kodierten musikalischen Informationen zu extrahieren.
Es wird zunächst ein voll funktionsfähiges Verfahren zur Digitalisierung von verschiedenen Formaten von lochplattenförmigen Medien vorgestellt.
Ebenso wird eine erste Implementierung für ein Verfahren zur Digitalisierung von Notenrollen eingeführt.
Die Anwendung gibt nach der Verarbeitung der Bilddaten der Medien eine standardisierte Midi-Datei aus.
Im weiteren werden die in die Anwendung integrierte Analyse- und Visualisierungstools, die in Zusammenarbeit mit den am Forschungsprojekt beteiligten Musikwissenschaftler:innen zur Beantwortung musikwissenschaftlicher Fragestellungen entwickelt wurden, demonstriert.

