\section{Einleitung}

Historische Medien wie Lochplatten und Klavierrollen sind die ersten kommerziell in größerer Stückzahl produzierten Toninformationsträger.
Sie wurden vom Ende des 19. Jahrhunderts bis zur frühen Mitte des 20. Jahrhunderts mit einer Vielzahl von Musikstücken und in einer fast ebensohohen Menge von Formaten produziert.
Die Medien sind die Zeugnisse einer Zeit als privater Musikkonsum erstmals einer breiteren Gruppe von Personen möglich wurde.
Neben vom Hersteller aus Kompositionen erstellten Medien existieren auch eine Vielzahl von sogenannten Player Rolls, Notenrollen die an einem speziell modifizierten Klavier von einem Pianisten eingespielt wurden.
Diese bilden die einzigen Aufnahmen einer Vielzahl von bekannten Pianisten aus einer Zeit in der Audioaufnahmen technisch noch nicht ausgereift für solche Zwecke waren.

Diese Faktoren machen diese historischen Toninformationsträger zu einer unschätzbaren Resource für die moderne Musikwissenschaftliche Forschung.\\
Während eine signifikante Menge, insbesondere von Notenrollen, bis heute über-dauert hat, wird sich der Zustand der Medien auf 
absehbare Zeit weiter verschlechtern und so zu weiteren irreversiblen Verlusten führen.
Um die Medien auch in Zukunft für die musikwissenschaftliche Forschung zugänglich zu machen und gleichzeitig weitere Schäden an den Medien durch die Nutzung zu minimieren bietet sich die Digitalisierung der selbigen an.
Dazu gehört zum einen die Erstellung von Hochauflösenden Bildern der Medien um diese möglichst originalgetreu abzubilden, zum anderen aber auch die inhaltliche Digitalisierung der in den Medien kodierten Toninformationen.
Verfahren zu diesem Zweck existieren, sind aber zu meist schlecht dokumentiert und nicht öffentlich zugänglich (siehe Abschnitt \ref{Forschungsstand}).

Das Musikinstrumentenmuseum der Universität Leipzig ist im Besitz einer signifikanten Sammlung von solchen historischen Toninformationsträgern.
Im Rahmen des vergangenen Forschungsprojektes TASTEN wurden von ca. 3000 der Notenrollen aus dieser Sammlung Scans erstellt.
Ziel des aktuellen Forschungsvorhabens DISKOS an der Forschungsstelle des Musikinstrumentenmuseums ist die Digitalisierung dieser Notenrollen und weiterer, plattenförmiger Toninformationsträger aus derselben Sammlung sowie die inhaltliche Analyse der digitalisierten Medien.
Zur Umsetzung dieser Ziele soll eine Anwendung entwickelt werden die Medien aller sich im Bestand des Musikinstrumentenmuseums befindlichen Formate verarbeiten kann.
Die Anwendung soll auch für die späteren inhaltlichen Analysen nutzbar sein.

Zu diesem Zweck wurde die in dieser Arbeit vorgestellte Anwendung entwickelt.
Sie ist in der Lage aus den Bilddaten von historischen Toninformationsträgern die darauf kodierten musikalischen Informationen zu extrahieren.
Es wird zunächst ein voll funktionsfähiges Verfahren zur Digitalisierung von verschiedenen Medien in Form von Lochplatten vorgestellt.
Auch eine erste Implementierung zur Verarbeitung von als Notenrollen vorliegenden Medien wird eingeführt.
Die Ausgabe der Anwendung bildet eine standartisierte Midi-Datei.
Im weiteren werden in die Anwendung integrierte Analyse- und Visualisierungstool die in Zusammenarbeit mit der am Forschungsprojekt beteildigten Musikwissenschaftler:innen zur beantwortung Musikwissenschaftlicher Fragestellungen entwickelt wurden, demonstriert.

