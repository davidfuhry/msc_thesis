\section{Fazit}

Ein Ziel des Forschungsprojektes DISKOS ist die Erstellung einer Anwendung zu bildbasierten Digitalisierung verschiedener Formate von historischen Toninformationsträgern mit dem Ziel diese wichtigen Zeitgeschichtlichen Zeugnisse zu präservieren und der Musikwissenschaftlichen Forschung zu öffnen.
Die Anwendung soll neben der reinen Digitalisierung auch die Möglichkeit bieten Analyse- und Visualsierungverfahren einzusetzen die zur Beantwortung der inhaltlichen Fragestellungen des Forschungsprojektes beitragen können.

In dieser Arbeit wurde die Implementierung dieser Anwendung in Form der Python Software \code{musicbox} vorgestellt.
Es wurde ein Verfahren demonstriert mit dem diese Anwendung aus Plattenbasierte Medien erfolgreich und vollständig digitale Midi Dateien erzeugen kann.
Ferner wurde auch ein grundlegendes Verfahren zur Digitalisierung von Medien in Form von Notenrollen vorgestellt, welche bereits für die Umwandlung einzelner Rollen nutzbar ist.
Auch wurden erste, in der Anwendung integrierte Analyse- und Visualsierungstools vorgestellt die zur Beantwortung Musikwissenschaftlicher Fragestellungen an den Medienbestand genutzt werden können.

Diese Arbeit konnte ein erfolgreiches Verfahren für die Digitalisierung historischer Toninformationsträger demonstrieren.
Mit der so entstandenen modular und erweiterbar konzipierten Anwendung wurde eine gute Grundlage geschaffen um in Zukunft auch die weiteren Fragestellungen des DISKOS Projektes beantworten zu können.