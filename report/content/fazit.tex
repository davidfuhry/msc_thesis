\section{Fazit}

Historische Toninformationsträger in Form von u.a. Lochplatten und Notenrollen stellen wertvolle, zeithistorische Zeugnisse aus einer Zeit in der privater Musikkonsum erstmals möglich wurde dar.
Im Rahmen der Forschung am Musikinstrumentenmuseum der Universität Leipzig sollen diese Medien präserviert und für die musikwissenschaftliche Forschung geöffnet werden.
Dazu gehört die softwarebasierte Erzeugung von Midi-Dateien aus Bildaufnahmen der Originalmedien.
Auch sollen verschiedene Analyse- und Visualsierungtools geschaffen werden die zur Beantwortung musikwissenschaftlicher Fragestellungen an die Medien beitragen können.

In dieser Arbeit wurde die Implementierung einer Anwendung zu diesem Zweck in Form der Python Software \code{musicbox} vorgestellt.
Es wurde ein Verfahren eingeführt mit dem diese Anwendung aus Lochplatten erfolgreich und vollständig digitale Midi-Dateien erzeugen kann.
Weiterhin wurde ein grundlegendes Verfahren zur Digitalisierung von Notenrollen vorgestellt, welches in seinen Grundfunktionen bereits für die Digitalisierung einzelner Medien nutzbar ist.
Abschließend wurden erste, in der Anwendung integrierte, Analyse- und Visualsierungstools beschrieben, die zur Beantwortung musikwissenschaftlicher Fragestellungen an den Medienbestand genutzt werden können.

Insgesamt konnte ein erfolgreiches Verfahren für die Digitalisierung historischer Toninformationsträger gezeigt werden.
Die dabei entstandene Softwareanwendung bietet durch ihre auf einfache Erweiterbarkeit ausgelegte Architektur eine solide Grundlage für die Verarbeitung weiterer Medienformate sowie die Umsetzung zusätzlicher Visualisierungs- und Analyseverfahren.
In der Zukunft ist die Erweiterung der Anwendung für diese Zwecke geplant.